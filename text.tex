\documentclass[12pt, a4paper, oneside]{article} 

\usepackage{czech} % nastavení češtiny
\usepackage[center]{caption} 
\usepackage[utf8]{inputenc}
\usepackage{wrapfig} % nastavení obtékání textu
\usepackage{graphicx,amsmath} % nastavení grafiky, matematiky
\usepackage{subfig} % více obrázků vedle sebe 
\usepackage{float}
\usepackage{amsmath}
\usepackage{amssymb}
\usepackage{bbding}
\usepackage{enumitem}
\usepackage{breakurl}
\usepackage{pdflscape}
%\usepackage{indentfirst}

\usepackage{tocloft} %přidá tečky do obsahu ke kapitolám /sekcím 
\renewcommand{\cftsecdotsep}{\cftdotsep}

\usepackage[bookmarksopen,colorlinks,plainpages=false,linkcolor=black,urlcolor=blue,citecolor=black,filecolor=black,menucolor=black,unicode=true]{hyperref}

\urlstyle{rm}
%bookmarksopen -- open up bookmark tree 
%colorlinks -- zbarví odkazy (implicitně orámovaný nezbarvený text)
%urlcolor -- barva odkazů (implicitně magenta) 
%linkcolor=black -- barva odkazů v obsahu (implicitně red)


\usepackage{listings}
\usepackage{color}
%\usepackage{minted}
\definecolor{lightgray}{RGB}{240,240,240}
\definecolor{darkgray}{rgb}{.4,.4,.4}
\definecolor{purple}{rgb}{0.65, 0.12, 0.82}
\definecolor{darkgreen}{RGB}{0,150,0}

%\usemintedstyle{perldoc}
%\newminted{js}{linenos=true, bgcolor=lightgray}

\lstdefinelanguage{JavaScript}{
  keywords={typeof, new, true, false, catch, function, return, null, catch, switch, var, if, in, while, do, else, case, break, for},
  keywordstyle=\color{blue}\bfseries,
  ndkeywords={class, export, boolean, throw, implements, import, this},
  ndkeywordstyle=\color{blue}\bfseries,
  identifierstyle=\color{black},
  sensitive=zr,
  comment=[l]{//},
  morecomment=[s]{/*}{*/},
  commentstyle=\color{darkgreen}\ttfamily,
  stringstyle=\color{red}\ttfamily,
  morestring=[b]',
  morestring=[b]"
}

\lstset{
   backgroundcolor=\color{lightgray},
   extendedchars=true,
   basicstyle=\footnotesize\ttfamily,
   showstringspaces=false,
   showspaces=false,
   numbers=left,
   numberstyle=\footnotesize,
   numbersep=9pt,
   tabsize=2,
   breaklines=true,
   showtabs=false,
   aboveskip=5mm,
   belowskip=7mm,
   captionpos=b
}

\renewcommand{\listingscaption}{Příklad}
\renewcommand{\listoflistingscaption}{Příklady}

% \usepackage{parskip} -- zapne americké odstavce v celé práci

\addtolength{\textwidth}{-2mm} 
\addtolength{\hoffset}{4mm}  % posun textu kvůli kroužkové vazbě  

\setlength{\intextsep}{5mm} % nastavení mezery okolo obrázků

% nastavení příkazu >\figcaption pro popis čehokoli, jako by to byly obrázky 
\makeatletter   
\newcommand\figcaption{\def\@captype{figure}\caption}
\makeatother

\renewcommand\refname{Literatura} 
%\def\bibname{PŘÍLOHA D: Reference}
%\renewcommand\bibname{PŘÍLOHA D: Reference}
% přejmenuje anglický název Reference na české Literatura


%\makeindex % příprava pro výrobu indexu (jestli ho chcete)

%%    VLNKA <fileinput>  KkSsVvZzOoUuAaIi        
% Defaultni  koncovka pro <fileinput> je  ".tex"
%FIXME: haze error
%\cstieon % Vypne chovani vlnky jako tvrde mezery v matematickem rezimu

%%%%%%%%%%%%%%%%%%%%%%%%%%%%%%%%%%%%%%%%%%%%%%%%%%%%%%%%%%%%%%%
%V PROSTŘEDÍ ROVNIC SE NESMÍ VYSKYTOVAT PRÁZDNÝ ŘÁDEK
%
%PROGRAMY VLNKA A CSINDEX SE MUSÍ SPUSTIT SAMOSTATNĚ
%%%%%%%%%%%%%%%%%%%%%%%%%%%%%%%%%%%%%%%%%%%%%%%%%%%%%%%%%%%%%%%

% definice příkazů 
\newcommand{\D}{\medskip \noindent} % nový odstavec v "americkém" formátování 
\newcommand{\B}{\textbf} %tučné písmo
\newcommand{\A}{\mathbf} %tučné písmo v matematickém režimu
\newcommand{\TO}{\ensuremath{\boldsymbol\Omega}} % tučný znak velké omega -- pro ohmy
\newcommand{\I}{\index}  % vytváří položku indexu (asi nepoužijete)
\newcommand{\Deg}[1][]{\ensuremath{{#1}^\circ}} % vysází značku stupně Celsia
\newcommand{\Def}{\footnotesize Definice: \normalsize}
\newcommand{\Pos}{\footnotesize Experiment: \normalsize}
\newcommand{\Odv}{\footnotesize Odvození: \normalsize}
\newcommand{\Vym}{\footnotesize Vymezení pojmu: \normalsize}
\newcommand{\Ob}{obrázek }
\newcommand{\It}{\textit}  % kurzíva
\newcommand{\M}{\mathrm}   % v prostředí rovnic nastaví normální písmo (místo kurzívy ) 
\newcommand{\F}{\footnotesize} % zmenšená velikost písma
\newcommand{\N}{\normalsize} % normální velikost písma
%\newcommand{\U}{\underline}  % podtržené písmo
\newcommand{\e}{\ensuremath} 
\newcommand{\Has}{\textcolor{green}{\CheckmarkBold}}
\newcommand{\NoHas}{\textcolor{red}{\XSolidBrush}}
% další příkaz se aplikuje, pouze, když jste v matematickém režimu

%\hyphenation{Pusť-me pla-tí hod-no-ty do-sa-dí-me za-da-né dal-ším}
% dělení slov, kdyby implicitní nevyhovovalo

\linespread{1.3} 
% řádkování 1,5x  
% použijete podle situace  

\unitlength=1mm % nastavení volby jednotek 

% konec hlavičky
%%%%%%%%%%%%%%%%%%%%%%%%%%%%%%%%%%%%%%%%%%%%%%%%%%%%%%%%%%%%%%%%%%%
%%%%%%%%%%%%%%%%%%%%%%%%%%%%%%%%%%%%%%%%%%%%%%%%%%%%%%%%%%%%%%%%%%%

\begin{document} % začátek textové části 

% titulní strana
\pagestyle{empty} % vynechá číslování
 
\voffset = -20mm % posun začátku textu výš
\enlargethispage{60mm} % zvětší oblast tisku pro tuto stránku   

\begin{center}
 
\Large \B{STŘEDOŠKOLSKÁ ODBORNÁ ČINNOST}

\vspace{60mm}

\Huge %\LARGE
\B{MultiROM} \\
\LARGE
\B{Nástroj pro instalaci více operačních systému na mobilní zařízení}
% na titulní straně může být stručnější, pokud je to potřeba  

\Large

\vspace{90mm}


\B{Vojtěch Boček} \\

\vspace{40mm}

\B{Brno 2014}


\end{center}

\newpage % konec titulní strany 
%%%%%%%%%%%%%%%%%%%%%%%%%%%%%%%%%%%%%%%%%%%%%%%%%%%%%%%%%%%%%%%%%%%%%%%%%%%

% vnitřní titulní strana
\voffset = -20mm % posun začátku textu výš
\enlargethispage{60mm} % zvětší oblast tisku pro tuto stránku   

\begin{center}

\Large \B{STŘEDOŠKOLSKÁ ODBORNÁ ČINNOST}  \\
\vspace{10mm}
 \normalsize 
\B{Obor SOČ: 18. Informatika}% číslo a název -- vyplníme spolu 

\vspace{45mm}

\Huge
\B{MultiROM} \\
\LARGE
\B{Nástroj pro instalaci více operačních systému na mobilní zařízení}
\end{center}
\large

\vspace{50mm}


\begin{tabbing}
\hspace{10mm} \= \hspace{30mm}  \=   \kill % nastavení zarážek 
  \> \B{Autor:}  \> \B{Vojtěch Boček}        \\[8mm] 
  \> \B{Škola:}   \> \B{SPŠ a~VOŠ technická, }     \\
  \>              \> \B{Sokolská 1, 602 00 Brno}    \\[8mm]
\end{tabbing}

\vspace{20mm}

\begin{center}
\B{Brno 2014}

\end{center}
\normalsize
%%%%%%%%%%%%%%%%%%%%%%%%%%%%%%%%%%%%%%%%%%%%%%%%%%%%%%%%%%%%%%%%%%%%%%%%%%%
\newpage  % Prohlášení o autorství  
\voffset = 0mm % posun začátku textu zpět

~ % musí to tu být, aby fungovala svislá mezera

\vspace{10mm}

\section*{Prohlášení}

Prohlašuji, že jsem svou práci vypracoval samostatně, použil jsem pouze podklady (literaturu, SW atd.) citované v~práci a~uvedené v~přiloženém seznamu a~postup při zpracování práce je v~souladu se zákonem č. 121/2000 Sb., o~právu autorském, o~právech souvisejících s~právem autorským a~o~změně některých zákonů (autorský zákon) v~platném znění. 
 
\vspace{20mm} 
 
\noindent V~Brně  dne: 6.3.2014 \hspace{50mm}                 podpis:   
 

%%%%%%%%%%%%%%%%%%%%%%%%%%%%%%%%%%%%%%%%%%%%%%%%%%%%%%%%%%%%%%%%%%%%%%%%%%%
\newpage   % Poděkování -- nepovinné 

~ % musí to tu být, aby fungovala svislá mezera
\vspace{100mm}

\section*{Poděkování}
Děkuji Jakubu Streitovi za rady, obětavou pomoc, velkou trpělivost a~podnětné připomínky poskytované během práce na tomto projektu, Martinu Vejnárovi za informace o~programátoru Shupito, panu profesorovi Mgr. Miroslavu Burdovi za velkou pomoc s~prací a~v~neposlední řadě Bc. Martinu Foučkovi za rady a~pomoc při práci s~Qt Frameworkem. Dále děkuji organizaci DDM Junior za poskytnutí podpory.
 

%%%%%%%%%%%%%%%%%%%%%%%%%%%%%%%%%%%%%%%%%%%%%%%%%%%%%%%%%%%%%%%%%%%%%%%%%%%
\newpage   % Anotace 
~ % musí to tu být, aby fungovala svislá mezera
\vspace{-20mm}

\section*{Anotace}

%MultiROM is one-of-a-kind multi-boot mod for Nexus 7. It can boot any Android ROM as well as other systems like Ubuntu Touch, Plasma Active, Bohdi Linux or WebOS port.Besides booting from device's internal memory, MultiROM can boot from USB drive connected to the device via OTG cable. The main part of MultiROM is a boot manager, which appears every time your device starts and lets you choose ROM to boot. You can see how it looks on the left image below and in gallery. ROMs are installed and managed via modified TWRP recovery. You can use standard ZIP files to install secondary Android ROMs, daily prebuilt image files to install Ubuntu Touch and MultiROM even has its own installer system, which can be used to ship other Linux-based systems.

Tato práce popisuje modifikaci pro mobilní zařízení s operačním systémem Google Android, která umožňuje instalaci více operačních systémů vedle sebe, podobně jako na PC. Může to být pouze několik různých verzí OS Google Android, ale i úplně jiné systémy - například právě se vyvíjejicí Ubuntu Touch a další.

Možnost provozovat více operačních systému na jedno zařízení je užitečná spíše pro pokročilé uživatel, podobně jako na stolních počítačích, nicméně počet aktivních uživatelů dokazuje, že zájem o tuto modifikaci rozhodně existuje.


%Tato práce popisuje komplexní sadu nástrojů pro vývoj a~ovládání libovolného zařízení schopného komunikovat po sériové lince nebo TCP socketu.

%Protože se nejedná o~jednoduchou aplikaci ani o~jednostranně zaměřenou sadu nástrojů, protože se navíc celá sada průběžně rozrůstá a~protože je rozsah a~záběr použití všech funkcí a~modulů této sady příliš velký, nelze ji stručně popsat v~omezeném prostoru anotace.

%Pro získání ucelenější představy o~tom, co vše lze pomocí sady Lorris dosáhnout, prosím nahlédněte do \hyperref[uvod]{úvodu práce}.

%Aplikace zastřešuje několik modulů. Každý modul je určen pro specifickou činnost -- parsování a zobrazování dat, programování mikročipů atd.

%Hlavním přínosem tohoto softwarového balíku je, že urychluje, zpřehledňuje a~hlavně výrazně zjednodušuje vývoj a~testování aplikací pro mikročipy, typicky programování a~řízení různých druhů robotů.

%\D \B{Klíčová slova:} analýza binárních dat, programování a~řízení robotů, vývoj pro mikrokontroléry, programování mikročipů

\addtolength{\textheight}{30mm} % prodlouží následující stránku

%%%%%%%%%%%%%%%%%%%%%%%%%%%%%%%%%%%%%%%%%%%%%%%%%%%%%%%%%%%%%%%%%%%%%%%%%%%
\newpage
\pagestyle{plain}

\setlength{\voffset}{-20mm} % posune text/obrázek na této stránce nahoru
\setcounter{page}{1}  % nastaví čítač stránek znovu od jedné

\tableofcontents  % vysází obsah

\addtolength{\textheight}{-30mm} % zkrátí následující stránku
%%%%%%%%%%%%%%%%%%%%%%%%%%%%%%%%%%%%%%%%%%%%%%%%%%%%%%%%%%%%%%%%%%%%%%%%%%%
\newpage
\setlength{\voffset}{0mm} % posune text/obrázek na této stránce, kam patří
%\pagestyle{headings} % znovu zapne číslování
\pagestyle{plain}

%
% Motivace 
\section*{Úvod}
\addcontentsline{toc}{section}{Úvod} % přidá položku úvod do obsahu
\label{uvod}



\section{Motivace}
Abych mohl vysvětlit důvod, proč je multi-boot na mobilních zařízeních užitečný, musím nejdříve upozornit na možnosti, které tato zařízení mají.

Tablety a telefony s platformou Google Android lze narozdíl od jiných systémů relativně snadno upravovat. Uživatelé na nich mohou získat přístup k tzv. superuživateli\footnote{Uživatel, který má práva přistupovat a měnit všechny části systému, je možné ho připodobnit k uživateli \It{Administrátor} v MS Windows. V Linuxu se jmenuje \It{root}.} a dále pak upravovat software na zařízení jakýmkoliv způsobem chtějí. Toto společně s faktem, že velká část OS Android má otevřené zdrojové kódy, vedlo ke vzniku obrovské komunity programátorů a nadšenců, kteří tyto zařízení různými způsoby upravují a vylepšují. Jedním z \uv{produktů} této komunity jsou celé upravené distribuce Androidu, tzv. ROM.

\subsection{Android ROM}
ROM lze přirovnat k distribucím Linuxu, jak je známe ze stolních počítačů. Jejich základem jsou obvykle zdrojové kódy z AOSP\cite{aosp}\footnote{\It{Android Open Source Project} - označení pro otevřenou část zrojových kódů OS Android}, které si autoři upravují podle svých představ - přidávají optimalizace pro zrychlení celého systému, přidávají další možnosti personalizace pro uživatele, mění prvky uživatelského rozhraní a mnoho dalšího. Na zařízeních, která již nejsou podporované výrobcem, mohou být ROM jediným způsobem, jak na ně přinést novější verze OS Android.

ROM často vydávají a spravují pouze jednotlivci, ale existují i projekty s velkým počtem vývojářů i uživatelů, například CyanogenMod\cite{CM}. Ten nedávno překonal hranici 10 milionů uživatelů, oficiálně podporuje přes 200 zařízení a je aktuálně největší upravenou ROM.

\subsection{Další operační systémy}
Kromě Androidu existují i další mobilní operační systémy, například Ubuntu Touch\cite{utouch}, Mozilla Firefox OS\cite{firefoxos} a další. Tyto systémy často použivají zařízení původně prodávaná s Androidem jako testovací, zejména kvůli jejich snadné dostupnosti a relativně nízké ceně. Pro tablet Google Nexus 7 dokonce existuje plná verze Linuxové distribuce Ubuntu, díky které je možné tento tablet po připojení klávesnice a myši používat jako netbook, s většinou programů které známe z PC.

\begin{figure}[H]
\begin{center}
\includegraphics[width=\textwidth]{img/n7_ubuntu.jpg}
\caption{Ubuntu na tabletu Nexus 7 s připojenou klávesnicí}
\label{split_img}
\end{center}
\end{figure}


\subsection{Více systémů na jednom zařízení}
Tyto telefony a tablety tedy ve většině případů (více o této problematice v příloze \uv{\nameref{sec:locked}}) nejsou uzamknuté na jediném, výrobcem vybraném systému a jeho verzi a dokáží podobně jako stolní počítače provozovat více typů a verzí operačních systémů. Chybí jim však možnost mít více systémů nainstalovaných zárověň (dále jen \It{multiboot}). Tato skutečnost pravděpodobně nevadí běžnému uživateli, nicméně pro zkušené uživatele a vývojáře je \It{multiboot} neocenitelnou funkcí. Mohou ho využít například při:

\begin{itemize}
    \item \B{vývoji aplikací pro Android} - programátoři aplikací pro Android často mají své zařízení odemknuté a systém který používají různými způsoby upravený, nicméně takové prostředí není příliš vhodné pro vývoj aplikací pro většinu uživatelů, která nic takového nepoužívá. S \It{multibootem} mohou mít jeden systém se svými úpravami pro běžné používání a druhý, neupravený, na kterém mohou bez obav testovat své aplikace. \\
    Při psaní aplikací je čas od času třeba vyzkoušet určitou funkcionalitu na jiné verzi OS Android, to je s \It{multibootem} také mnohem jednodušší - není třeba přepsat aktuální systém na zařízení, stačí nainstalovat požadovanou verzi OS jako další operační systém.

    \item \B{zkoušení jiných ROM} - ROM vytvořené komunitou mají v některých případech nové a velmi zajímavé funkce, nicméně je často vhodné vyzkoušet si danou ROM před tím, než uživatel projde zdlouhavým procesem obnovování všech svých dat do této nové ROM. S \It{multibootem} je možné novou ROM jen nainstalovat jako další systém bez toho, aby se smazal ten původní a v případě, že nová ROM uživateli nesedne, může ji kdykoliv smazat. Jeho původní systém se všemi daty je stále neporušený.\\
    Toto platí i pro jiné operační systémy - příkladem budiž Ubuntu Touch od firmy Canonical, nový operační systém s ovládáním značně odlišným od Androidu. Řada uživatelů by si tento OS ráda zkusila, nicméně Ubuntu Touch je stále ve vývoji a není ještě zcela schopný plně nahradit Android. Bez \It{multibootu} by museli museli smazat svůj hlavní OS, což řadu uživatelů odradí.

    \item \B{potřebě využívat aplikace z více operačních systémů} - jako příklad opět použiji Ubuntu Touch, který bych jako nadšenec chtěl používat, nicméně v práci potřebuji určitou aplikaci, která je kompatiblní pouze s OS Android. S \It{multibootem} si nemusím vybírat, stačí přepínat mezi těmito OS podle aktuální situace. Toto použití můžeme pozorovat i na PC, kde má řada uživatelů Linuxu současně nainstalovaný i MS Windows, protože potřebují používat nějakou proprietární aplikaci, která není s Linuxem kompatibilní.
% TODO: doplnit
\end{itemize}

\noindent \It{Multiboot} je tedy užitečný i na přenosných zařízeních, což, jak se přesvědčíte dále v této práci, potvrzuje i počet uživatelů MultiROM.

\section{MultiROM}
MultiROM je modifikace pro zařízení původně s OS Android, která umožňuje používat více operačních systémů zárověň na jednom zařízení. Skládá se ze čtyř částí, které budou dalé důkladně popsány v samostatných kapitolách:

\begin{enumerate}
    \item \B{boot manager} - hlavní část tohoto projektu. Zobrazuje se v průběhu startu zařízení, nechá uživatele vybrat, který operační systém chce spustit a provede akce nutné ke spuštění tohoto systému.
    \item \B{upravená recovery} - tato část se stará o snadnou instalaci a správu nainstalovaných systémů.
    \item \B{kexec-hardboot patch} - modifikace Linuxového jádra, která umožňuje nastartovat jiné jádro z již běžícího, tedy ukončit aktuální operační systém a spustit nový.
    \item \B{aplikace pro Android} - umí rychle a jednoduše nainstalovat všechny potřebné části MultiROM a aktualizovat je a do omezené míry spravovat nainstalované systémy a také do nich přepnout přímo z Androidu (bez nutnosti restartovat zařízení a až poté vybrat systém v boot manageru).
\end{enumerate}

MultiROM je svobodný software, distribuovaný pod licencí GNU GPLv3 (boot manager, recovery a Android aplikace) a GNU GPLv2 (kexec-hardboot patch). Zdrojové kódy jsou k dispozici v repozitářích na serveru GitHub:
\begin{itemize}
    \item \B{Boot manager:} \url{http://github.com/Tasssadar/multirom}
    % TODO fixme
    \item \B{Recovery:} \url{http://github.com/Tasssadar/}
    \item \B{Android aplikace}: \url{http://github.com/Tasssadar/MultiROMMgr}
\end{itemize}

\subsection{XDA Developers fóra}
Server \url{http://xda-developers.com} je největším webem zabývajícím se upravováním Android (ale i jiných) zařízení. Jeho hlavní částí je fórum, kde vývojáři a nadšenci vydávají své produkty. Právě zde lze nalézt nejvíce různých druhů ROM a dalších modifikací.

Hlavním způsobem, jak získává MultiROM nové uživatele a já komunikuji s existujícími jsou právě vlákna na XDA fóru. Obsahují popis MultiROM, návod na instalaci, návod k používání, seznam změn a odkazy ke stažení. Každé zařízení má svě vlastní vlákno:

\begin{itemize}
\item Nexus 7 (2012):\\ \url{http://forum.xda-developers.com/showthread.php?t=2011403}
\item Nexus 7 (2013):\\ \url{http://forum.xda-developers.com/showthread.php?t=2457063}
\item Nexus 4: \url{http://forum.xda-developers.com/showthread.php?p=46223377}
\item Nexus 5: \url{http://forum.xda-developers.com/showthread.php?t=2571011}
\end{itemize}

\subsection{Kompatibilní zařízení}
Platforma ARM bohužel stále není standardizovaná alespoň do takové míry jako x86, kterou známe z PC, a proto není možné vytvořit tuto nízkouúrovňovou modifikaci tak, aby fungovala se všemi Android telefony a tablety. V současnosti oficiálně podporuji tato zařízení:

\begin{enumerate}
    \item Telefon \B{Google Nexus 4}
    \item Telefon \B{Google Nexus 5}
    \item Tablet \B{Google Nexus 7 (verze z roku 2012)}
    \item Tablet \B{Google Nexus 7 (verze z roku 2013)}
\end{enumerate}

Seznam není příliš dlouhý, zejména protože s každým dalším podporovaným zařízením značně narůstá čas, který musím věnovat testování a protože jen velmi nerad podporuji zařízení, které nevlastním - jak jsem se přesvědčil u Nexusu 4, značně to omezuje schopnost řádně otestovat nové funkce a úpravy, protože musím spoléhat na některého ze zkušenějších uživatelů MultiROM, aby vše řádně otestoval, případně pomohl s řešením problému.

MultiROM má ale svobodný zdrojový kód, takže kdokoliv může přidat podporu pro další zařízení, což se také stalo - existují neoficiální verze například pro telefony HTC One, Samsung Galaxy S3 nebo Sony Xperia M. Verze pro Nexus 4 byla původně také upravena jiným členem komunity, já jsem ji převzal a poskytuji pro ni podporu.

\subsection{Požadavky}
MultiROM mění části systému, zařízení tedy musí být odemknuté. Všechny oficálně podporované přístroje patří do řady Nexus, jsou tedy \uv{volně odemykatelné} (více o této problematice v příloze \uv{\nameref{sec:locked}}).

MultiROM aplikace pro Android vyžaduje, aby bežela na systému s přístupem k \It{superuživateli}, protože potřebuje přistupovat do části systému které nejsou běžným aplikacím přístupné.

Naprostá většina potenciálních uživatelů MultiROM již tyto požadavky splňuje, situace kdy uživatel odemyká své zařízení jen kvůli MultiROM není příliš častá.

% TODO web, repo

\section{Boot manager}
Hlavní část MultiROM, která se spustí okamžitě po startu Linuxového jádra, na kterém Android běží, ještě před tím než se začne spouštět samotný Android. Uživatel v ní může vybrat, který systém chce použít a boot manager poté provede akce potřebné ke spuštění vybraného systému.

%TODO screenshot

\section{Upravená recovery}
\subsection{Obecné představení pojmu}
Pojem \uv{recovery} u zařízení s OS Android označuje velmi malý (v řádech několika megabajtů), oddělný systém původně určení pro instalaci aktualizací OS od výrobce, případně reinstalaci systému v případě jeho porušení. Recovery je obvykle možné spustit vypnutím zařízení a následným podržením určité kombinace tlačítek (např. vypínací tlačítko a tlačítko pro zvýšení hlasitosti zároveň). S podobným nápadem se můžeme setkat i u PC, kdy je na \uv{Oddílu pro obnovu} kopie operačního systému, ze které lze obnovit ten hlavní.

Komunita možnosti recovery značně rozšířila a vylepšila a v současné době existuje několik konkurenčních upravených recovery. Právě přes upravené recovery se instalují ROM vydané členy komunity, umí zálohovat celé zařízení a mají řadu dalších funkcí.

\subsection{TeamWin Recovery Project}
% TODO: screenshot
TeamWin Recovery Project (dále jen TWRP) je jedna z upravených recovery a pravděpodobně má nejvíce funkcí. Má grafické rozhraní plně podporující dotykový displej, podporuje témata (celé GUI je napsané pomocí XML souborů a je možné ho jakkoliv upravit), dokáže instalovat ROM, dělat zálohy, mazat diskové oddíly, obsahuje jednoduchý terminál a další. 

% TODO: doplnit

\subsection{Jak MultiROM využívá TWRP}
TWRP je svobodným softwarem a kdokoliv do ní může přispívat a používat její zdrojové kódy, což mi umožnilo použít ji ke správě systémů nainstalovaných v MultiROM. TWRP jsem za tímto účelem značně upravil a při práci na ní jsem vytvořil i několiv oprav a úprav, které bylo možné zaslat i do hlavního vývojového stromu TWRP.

Pomocí upravené TWRP se tedy instalují a spravují vedlejší systémy a upravuje se v ní nastavení MultiROM. Za tímto účelem jsem vytvořil v TWRP další položku menu, pod kterou jsou všechny tyto funkce k dispozici, rozhraní již existujících částí recovery je změněné jen minimálně, takže návyky uživatelů z původní TWRP stále platí.

Grafické rozhraní TWRP mi umožnilo vytvořit relativně snadný a rychlý způsob instalace vedlejších ROM - stačí zvolit typ systému, umístění instalace (kromě interní paměti dokáže MultiROM bootovat i z USB flash disku) a instalační soubory, o zbytek se postará recovery.

\subsection {Podporované formáty instalačních souborů}
\B{Android ROM} se instalují pomocí ZIP souborů obsahujících systém a script, který recovery spustí. Tento způsob, pojmenovaný \It{Edify script}, původně používal Google pro instalaci aktualizací přes recovery, nicméně komunita ho velmi rychle adoptovala a používá se k instalaci naprosté většiny Android ROM.

\B{Ubuntu Touch} má dva způsoby instalace - první, starší, používá dva ZIP soubory s \It{Edify scriptem}, moje upravená TWRP umožňuje uživateli vybrat oba soubory na začátku instalace (nemusí je instalovat postupně) a provede během instalace všechny změny potřebné pro běh s MultiROM. Druhý způsob je nový a používá oddělené datové soubory a soubor s příkazy, který zpracovává script v recovery - tento způsob instalace je třeba provádět přes aplikaci pro Android, protože pro stažení instalačních souborů je třeba parsovat určitá metadata (který datový soubor patří ke které verzi systému, ...).

\B{Ubuntu Desktop} pro Nexus 7 (2012) je distribuován jako obraz diskového oddílu. Moje upravená TWRP ho dokáže rozbalit, nainstalovat a upravit tak, aby běžel v prostředí MultiROM. 

\B{MultiROM Instalátor} - nový typ instalačních souborů, který jsem vytvořil speciálně pro MultiROM. Jedná se o nekomprimovaný ZIP archiv, který obsahuje datové soubory (archivy .tar.gz), soubor s informacemi pro MultiROM (podporované souborové systémy, příkazová řádka jádra, ...) a může obsahovat i scripty v jazyce bash, které mohou být spuštěny v různých bodech instalace a udělat, co je pro daný systém potřeba. Více o instalátoru najdete v příloze \uv{\nameref{sec:installer}}.

\section {Princip multibootu}
MultiROM instaluje všechny vedlejší operační systémy do podsložky na \uv{kartě SD}\footnote{Řada zařízení již fyzickou SD kartou nedisponuje, \uv{SD kartou} se tedy označuje část vnitřní paměti přístroje, která kartu nahrazuje}. Každý systém má svoji vlastní podsložku, která je později funguje jako kořen tohoto systému místo původního, hlavního kořene. Další postup se liší podle typu systému.

\subsection{Android}
OS Android má proces startu systému a uspořádání systému souborů odlišné od běžných Linuxových distribucí. Jádro systému si ihned po startu rozbalí archiv souborů do operační paměti a použije ho jako kořen souborového systému. Tento archiv je běžně označován jako \It{init ramdisk}, zkráceně \It{initrd}. \It{Initrd} obsahuje první program, který se spustí po startu jádra systému, tvz. \It{init}. Tento má za úkol inicializovat zařízení v telefonu, připojit všechny \uv{diskové oddíly} a spustit z nich zbytek Androidu, k čemuž používá \It{scripty}, které čte a vykonává akce v nich definované.

Před ním než budu pokračovat, musím upozornit na to, že Android používá tři hlavní diskové oddíly: \B{/data}, pro aplikace a uživatelská data, \B{/system}, pro nainstalované neměnné části systému a \B{/cache} pro dočasné soubory. Nahrazení všech těchto složek jinými prakticky znamená spuštění jiného systému.

MultiROM \B{nahrazuje} \It{init} v \It{initrd}, je tedy prvním programem, který běží po startu jádra. Poté, co uživatel vybere systém, který chce spustit, boot manager provede následující proces:
%buďto použije kexec k restartování zařízení s jádrem vybraného systému nebo pokračuje dále s již spuštěným jádrem, které se tedy sdílí mezi hlavní a vedlejší ROM (možné nastavit při instalaci ROM). 

\begin{enumerate}
%todo 1a/1b
    \item Použije kexec k \B{restartování zařízení s jádrem vybraného systému} nebo \B{pokračuje dále se stejným jádrem}, které se tedy sdílí mezi hlavní a vedlejší ROM (možné nastavit při instalaci ROM).
    \item Přípojí podsložky vybrané ROM z SD karty na umístění v kořenu souborového systému, tedy /data, /system a /cache.
    \item Přepíše scripty původní ROM těmi z uživatelem vybrané ROM.
    \item \B{Upraví scripty tak, aby již nepřipojovali původní diskové oddíly na /data, /system a /cache} - přepsali by složky vedlejší ROM, které boot manager již připojil.
    \item Boot manager se ukončí a \B{spustí místo sebe \It{init} z uživatelem vybrané ROM}, který spustí tuto ROM ze složek, které boot manager připojil na /data, /system a /cache.
\end{enumerate}
%TODO obrázek

Multiboot systému Android je poměrně přímočarý a MultiROM ho zvládne provést sama, není třeba danou ROM nijak modifikovat aby byla s MultiROM kompatibilní.

\subsection{Ostatní systémy}
MultiROM zvládne nabootovat jakýkoliv systém založený na Linuxovém zařízení (žádný jiný druh není k dispozici), ale je potřeba tyto systémy modifikovat, aby dokázali použít podsložku SD karty jako kořen systému souborů. Boot manager spouští tento typ systémů následujícím způsobem:
\begin{enumerate}
%todo 1a/1b
    \item Použije kexec k \B{restartování zařízení s jádrem a \It{initrd} vybraného systému}. Nastaví přitom do příkazové řádky jádra\footnote{Několik textových parametrů jádra, např. "\verb-console=tty1 root=/dev/sda1 rootfstype=ext4 debug-"} \B{cestu do složky s kořenem spouštěného systému}
    \item Po restartu již MultiROM systém \B{nijak neovlivňuje}. Je na systému samotném, aby jako kořenový systém připojil složku, která mu byla předána v příkazové řádce jádra.
\end{enumerate}

Tento typ systémů tedy musí byt upraven tak, aby byl schopný připojit složku předanou v příkazové řádce jádra jako kořen souborového systému. Jedná se obvykle o poměrně nenáročnou úpravu jednoho ze scriptů programu \It{init}, které většina OS založených na Linuxu má (ale nejsou všechny stejné, proto není možné upravovat je z MultiROM). V případě Ubuntu Touch dokáže recovery při instalaci systém patřičně upravit, je tedy možné použít původní, neupravené instalační soubory.

\newpage
\section*{Závěr}
\addcontentsline{toc}{section}{Závěr}



%%%%%%%%%%%%%%%%%%%%%%%%%%%%%%%%%%%%%%%%%%%%%%%%%%%%%%%%%%%%%%%%%%%%%%%%%%%
\newpage
\section*{PŘÍLOHA A:}
\section*{Uzamykání zařízení pouze na určitý systém}
\addcontentsline{toc}{section}{PŘÍLOHA A: Uzamykání zařízení pouze na určitý systém}
\label{sec:locked}
I když zdrojové kódy OS Android jsou z velké části otevřené, mezi výrobci panuje trend uzamykaní zařízení tak, aby mohli uživatelé používat pouze systém vydaný výrobcem, ať už v zájmu bezpečnosti, zachování určitých vlastností systému (např. předinstalované aplikace často obtěžující uživatele) nebo z prostého nepochopení trhu. V současnosti existuje několik typů uzamknutí zařízení:

\begin{enumerate}
    \item \B{Volně odemykatelné} - uživatel si je může sám odemknout, bez žádných nevýhod oproti uzamknutým zařízením. Tento typ je nejvíce ojedinělý, příkladem jsou zařízení z řady \It{Nexus} vydávané pod záštitou firmy Google.
    
    \item \B{Odemykatelné za registraci} - výrobce provozuje web, který po zadání sériového čísla zařízení vydá odemykací kód. Takto si vytvoří databázi odemknutých zařízení, kterou poté může využít k zamítnutí reklamací (podobné chování nemusí být v souladu s legislativou).
    
    Někteří výrobci při tomto odemknutí smažou klíče pro placený obsah (tzv. DRM\footnote{\It{Digital Rights Management} - systémy navrhnuté pro zabránění nelegálního kopírování digitálního obsahu - hudby, filmů, her apod.}), protože je na odemknutých systémech podle jejich názoru větší nebezpečí nelegálního kopírování obsahu chráněného pomocí DRM.
    
    \item \B{\uv{Edice pro vývojáře}} - výrobce kromě standardní varianty zařízení vydávají verzi, kterou je možné odemknout, buďto za registraci nebo volně. Tyto edice jsou často bezdůvodně dražší anebo nejsou dostupné ve všech zemích, ve kterých je možné koupit standardní variantu. Označení \uv{Developer edition}, které výrobci používají, je v tomto případě mírně zavádějící - zařízení je možné pouze odemknout a kromě této skutečnosti není o nic více přispůsobené vývojářům (části s uzavřeným zdrojovým kódem jsou stále nedostupné a hardware telefonu je rovněž stejný).

    \item \B{Bez možnosti odemknutí} - výrobce neposkytuje žádný způsob jak zařízení odemknout. V naprosté většině případů ale komunita překoná toto omezení a existuje více nebo méně pohodlná možnost neoficiálního odemknutí. Někteří výrobci tomuto odporují a dokonce přidávají vlastnosti kontrolující integritu systému a sledující, zda nebylo uzamknutí prolomeno. Tyto ochrany bývají často také překonány.
    
\end{enumerate}

\newpage
\section*{PŘÍLOHA B:}
\section*{MultiROM instalátor}
\addcontentsline{toc}{section}{PŘÍLOHA B: MultiROM instalátor}
\label{sec:installer}


\newpage
 \section*{PŘÍLOHA E:}
 \begin{thebibliography}{99}
\addcontentsline{toc}{section}{PŘÍLOHA E: Reference}
 %% 99 znamená, že maximální délka čísla literatury jsou dva znaky
% seznam samozřejmě změníte podle svého, tohle je pouze ukázka formátování

    \bibitem{aosp} \It{Android Open Source Project} \\
    \url{http://source.android.com/}\\
    (Stav ke dni 19.\,1.\,2014)

    \bibitem{CM} \It{CyanogenMod} \\
    \url{http://www.cyanogenmod.org/}\\
    (Stav ke dni 19.\,1.\,2014)

    \bibitem{utouch} \It{Ubuntu Touch} - Ubuntu on phones \\
    \url{http://www.ubuntu.com/phone}\\
    (Stav ke dni 21.\,1.\,2014)

    \bibitem{firefoxos} \It{Firefox OS}\\
    \url{http://www.mozilla.org/cs/firefox/os/}\\
    (Stav ke dni 21.\,1.\,2014)


\end{thebibliography}

\newpage
\section*{PŘÍLOHA G:}
~
\addcontentsline{toc}{section}{PŘÍLOHA G: Seznam obrázků}
\listoffigures   % seznam obrázkù 

\end{document}
